\exercise

Describe the Snow Plow algorithm, comment in which sorting algorithm can be used
and what is the advantage possibly induced in that sorting algorithm. Finally
simulate its working over the sequence 3, 10, 1, 6, 2, 5, 3, by showing the
sorted blocks it forms with a memory capable to store $M = 2$ integers.

\solution

\begin{algorithm}[b]
\caption{Snow Plow algorithm}\label{alg:snowplow}
\begin{algorithmic}[1]
\Require $\mathcal{U}$ unsorted array of $M$ items
\State Build a min-heap $\mathcal{H}$ over $\mathcal{U}$'s items
\State $\mathcal{U} \gets \emptyset$
\While{$\mathcal{H} \neq \emptyset$}
  \State $min \gets \min(\mathcal{H})$
  \State Write $min$ in the output run
  \State $next \gets$ next item from input sequence
  \If{$next < min$}
    \State Write $next$ in $\mathcal{U}$
  \Else
    \State Insert $next$ in $\mathcal{H}$
  \EndIf
\EndWhile
\end{algorithmic}
\end{algorithm}
%
The Snow Plow (\autoref{alg:snowplow}) creates an heap $\mathcal{H}$ over the
(unsorted) elements present in memory. Then it stores in the output run the
minimum of $\mathcal{H}$ ($min$) and loads in memory the next element of the
input sequence ($next$). If $next$ is less than $min$, it will be stored in an
unsorted auxiliary array $\mathcal{U}$, else it will be inserted in
$\mathcal{H}$. This will be repeated until $\mathcal{H}$ is empty. The output
run is still guaranteed to be sorted, and, exploiting the random distribution of
the input sequence, it will be also larger than the actual size of the memory,
since every new element of the sequence will be greater than the last minimum
with a probability of $\sfrac{1}{2}$. This method virtually increases the size
of $M$ by a factor of 2 (on average), thus lowering the I/O complexity of the
mergesort from $O\big(\frac{n}{B}\log_2\frac{n}{M}\big)$ to
$O\big(\frac{n}{B}\log_2\frac{n}{2M}\big)$ and saving 1 pass over the data.

Given the sequence 3, 10, 1, 6, 2, 5, 3, and a memory size of $M = 2$, we start
the algorithm with $\mathcal{U} = [3, 10]$ (the first 2 elements of the
sequence). Then
%
\begin{description}

  \item[Run 1.] We build $\mathcal{H}$ over $\mathcal{U}$, obtaining
  $\mathcal{H} = [3, 10]$.
  %
  \begin{enumerate}

    \item $min = 3,\ next = 1 < min \Longrightarrow \mathcal{H} = [10],\
    \mathcal{U} = [1]$

    \item $min = 10,\ next = 6 < min \Longrightarrow \mathcal{H} = [\,],\
    \mathcal{U} = [1, 6]$

  \end{enumerate}

  \item[Run 2.] We build $\mathcal{H}$ over $\mathcal{U}$, obtaining
  $\mathcal{H} = [1, 6]$.
  %
  \begin{enumerate}

    \item $min = 1,\ next = 2 \ge min \Longrightarrow \mathcal{H} = [2, 6],\
    \mathcal{U} = [\,]$

    \item $min = 2,\ next = 5 \ge min \Longrightarrow \mathcal{H} = [5, 6],\
    \mathcal{U} = [\,]$

    \item $min = 5,\ next = 3 < min \Longrightarrow \mathcal{H} = [6],\
    \mathcal{U} = [3]$

    \item $min = 6$, no input.

  \end{enumerate}

  \item[Run 3.] This run will only output 3, the last element of $\mathcal{U}$.

\end{description}
