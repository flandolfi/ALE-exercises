\exercise

Let us given a computational model with $D$ disks, memory $M$, and page size
$B$:
%
\begin{enumerate}

  \item Show the I/O-complexity of multi-way mergesort (for $D = 1$) and discuss
  what is worse in terms of I/O-complexity between either reducing the memory
  from $M$ to $\sqrt{M}$ or reducing the block size from $B$ to $\sqrt{B}$.

  \item Describe the technique of Disk Striping and show the I/O-complexity of
  the multi-way mergesort when applies it on $D$ disks.

\end{enumerate}

\solution

\begin{enumerate}

  \item The the I/O-complexity of the multi-way mergesort is $O\Big(\frac{n}{B}
  \log_{\frac{M}{B}} \frac{n}{M}\Big)$. Reducing the page size to $\sqrt{B}$ we
  obtain an I/O-complexity of $O\Big(\frac{n}{\sqrt{B}}
  \log_{\frac{M}{\sqrt{B}}} \frac{n}{M}\Big)$, which is better than reducing the
  memory to $\sqrt{M}$, obtaining a complexity of $O\Big(\frac{n}{B}
  \log_{\frac{\sqrt{M}}{B}} \frac{n}{\sqrt{M}}\Big)$, since the latter has a
  lower logarithm base and produces (asymptotically) worse results.

  \item The Disk Striping technique consists of looking at the $D$ disks as one
  single disk of page size $B' = DB$. The complexity of the multi-way mergesort
  becomes $O\Big(\frac{n}{B'} \log_{\frac{M}{B'}} \frac{n}{M}\Big) =
  O\Big(\frac{n}{DB} \log_{\frac{M}{DB}} \frac{n}{M}\Big)$, which is $1 -
  \log_{\frac{M}{B}} D$ slower than the optimal bound, causing the Disk Striping
  technique to be less and less efficient as the number of disks increases.

\end{enumerate}
