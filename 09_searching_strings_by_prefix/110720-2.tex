\exercise

Describe the interpolation-search data structure over the set of items $$S = \{
2, 3, 4, 9, 10, 18, 20, 21, 28, 30, 32, 36 \}.$$ Provide also the time bound for
the interpolation search of an integer $y$, and comment the scenario for which
this approach is better than the classic binary search.

\solution

We first divide the range $[S_1, S_n]$ in $n = |S| = 12$ bins $B_1, \dots, B_n$
of (approximately) equal size $b$, i.e. $$b = \left\lceil\frac{S_n - S_1 +
1}{n}\right\rceil = \left\lceil\frac{36 - 2 + 1}{1}\right\rceil = 3,$$ such that
$B_i = \{ s \in S \mid S_1 + (i - 1) \cdot b \le s < S_1 + i \cdot b \}$.  $S$
will be \emph{logically} partitioned as follows
%
\begin{center}
  \begin{tabular}{c|c|c|c|c|c|c|c|c|c|c|c}
    $B_1$ & $B_2$ & $B_3$ & $B_4$ & $B_5$ & $B_6$ & $B_7$ & $B_8$ & $B_9$ & $B_{10}$ & $B_{11}$ & $B_{12}$ \\
    2, 3, 4 & -- & 9, 10 & -- & -- & 18 & 20, 21 & -- & 28 & 30 & 32 & 36 \\
  \end{tabular}\ .
\end{center}
%
We then create an array $I$ of size $n$ such that each $I_i$ points to the first
and last elements of $B_i$ in $S$. When we want to search for an element $y$ in
$S$, we first compute its bin index $$j = \left\lfloor\frac{y -
S_1}{n}\right\rfloor + 1 ,$$ then we perform a binary search on the subset $S[l,
r]$, where $(l, r) = I_j$. The time complexity of this search is $O(\log
\Delta)$, where $\Delta$ is defined as $$\Delta = \frac{\max_{i = 2, \dots, n}
S_i - S_{i-1}}{\min_{i = 2, \dots, n} S_i - S_{i-1}},$$ so this algorithm
performs better than the binary search when the elements are uniformly
distributed. In the worst case, when all the elements in $S$ ends up in the same
bin, this algorithm has the same time complexity of the binary search (but using
an extra $O(n)$ space, while the binary search is in-place).
